\documentclass[a4paper]{article}
\usepackage[14pt]{extsizes} % для того чтобы задать нестандартный 14-ый размер шрифта
\usepackage[utf8]{inputenc}
\usepackage[russian]{babel}
\usepackage{setspace,amsmath}
\usepackage[left=20mm, top=15mm, right=15mm, bottom=15mm, nohead, footskip=10mm]{geometry} 
 
\begin{document} 
 
\begin{center}
\hfill \break
\large{МИНИСТЕРСТВО НАУКИ И ВЫСШЕГО ОБРАЗОВАНИЯ РОССИЙСКОЙ ФЕДЕРАЦИИ}\\
\footnotesize{Федеральное государственное автономное образовательное учреждение высшего образования}\\ 
\footnotesize{«Дальневосточный федеральный университет»}\\
\small{\textbf{(ДВФУ)}}\\
\hfill \break
\normalsize{ИНСТИТУТ МАТЕМАТИКИ И КОМПЬЮТЕРНЫХ ТЕХНОЛОГИЙ}\\
 \hfill \break
\normalsize{Департамент математического и компьютерного моделирования}\\
\hfill\break
\hfill \break
\hfill \break
\hfill \break
\large{Эффективная длинная арифметика}\\
\hfill \break
\hfill \break
\hfill \break
\normalsize{Доклад\\
\hfill \break
Направление подготовки 09.03.03 Прикладная информатика\\
\hfill \break
Профиль «Приладная информатика в компьютерном дизайне»}\\
\hfill \break
\hfill \break
\end{center}
 
\normalsize{} \hfill \break
\hfill \break
 
\normalsize{ 
\begin{tabular}{cccc}
Обучающийся & \underline{\hspace{3cm}} & &И.С. Щербак \\\\
Руководитель & \underline{\hspace{3cm}}& доцент ИМКТ &А.С. Кленин \\\\
\end{tabular}
}\\
\hfill \break
\hfill \break
\begin{center} Владивосток 2022 \end{center}
\thispagestyle{empty} 
 
\newpage
     
    \tableofcontents 
\newpage

\section{Введение}
\subsection{Информация}
Длинная арифметика - набор алгоритмов для поразрядной работы с числами произвольной длины. Она применяется как с относительно небольшими числами, превышающими ограничения типа long long в несколько раз, так и с по-настоящему большими числами (чаще всего до 10100000.). Для работы с “длинными” числами их разбивают на разряды.
\\Многие языки (Java, Ruby, Python) имеют встроенную поддержку длинной арифметики, что в разы может сократить время написания программы.
\subsection{История}
Первый бизнес-компьютер IBM 702 (цифровой компьютер на базе лапвовых компьютеров первого поколения) в середине 1950-х годов полностью реализовывал целочисленную арифметику в аппаратных срествах для строк цифр любой длинны от 1 до 511 цифр. Самая раняя широко распрастранненая программная реализация арифметики произвольной точности была реализована в Maclisp. 
\\Рання широко распространненная реализация была доступна через IMB 1620 (1959-1970 годов). Эта машина была десятичной, которая использовала дискретные транзисторы, но имела аппаратное обеспечение (которое использовало таблицы поиска) для выполнения целочисленной арифметики над строками цифр длиной от двух до любой доступной памяти. Для арифметики с плавающей точкой мантиса была ограничена сотней цифр или меньше, а показатель степени был ограничен только двумя цифрами. 
\newpage

\section{Описание}
\subsection{Структура данных}
На вход подаются два длинных числа. Далее, взависимости от знака, с ним выполняются разничные операции (сложение, вычитания, умножение, деление, возведение в степень).

\subsubsection{Сложение}
\\Разберем случай, когда складываются два числа. Процесс будет аналогичен способу сложения в столбик.
\\Для начала сравниваем длинну двух чисел. Если они разной длинны, то последние цифры длинного числа просто переносим к числу сумме. Далее берем элементы (цифры) с повторяющимися индексами с конца и складываем их. 
\\ \textbf{Пример}
\\ \textbf{Первое число} 1234567890987654321
\\ \textbf{Второе число} 123456789
\\ \textbf{Результат} 1234567891111111110

\subsubsection{Вычитание}
\\Для начала сравниваются два числа. Выбирается большее из них. И из большего вычитаем меньшее. Так же столбиком, то есть из последнего символа большего числа вычитаем последний символ меньшего числа.
\\ \textbf{Пример}
\\ \textbf{Первое число} 1234567890987654321
\\ \textbf{Второе число} 123456789
\\ \textbf{Результат} 1234567890864197532

\subsubsection{Умножение}
\\Для того, чтобы выполнить усножение, берутся элементы и умножаются друг на друга. Начиная с первого и заканчивая последним.
\\ \textbf{Пример}
\\ \textbf{Первое число} 1234567890987654321
\\ \textbf{Второе число} 123456789
\\ \textbf{Результат} 152415787623837841112635269

\subsubsection{Деление}
\\
\\ \textbf{Пример}
\\ \textbf{Первое число} 1234567890987654321
\\ \textbf{Второе число} 123456789
\\ \textbf{Результат}10000000008
\newpage

\section{Формальная постановка задачи}
\\Нужно написать программу, которая будет принимать на вход различные длинные числа и выполнять с ними операции (сложение, вычитание, умножение, деление).
\newpage

\section{Реализация}

\subsection{Тесты}
\\ \textbf{Тест 1}
\\ \textbf{Первое число} 1234567890987654321
\\ \textbf{Знак операции} +
\\ \textbf{Второе число} 123456789
\\ \textbf{Результат} 1234567891111111110
\\
\\ \textbf{Тест 2}
\\ \textbf{Первое число} 1234567890987654321
\\ \textbf{Знак операции} -
\\ \textbf{Второе число} 123456789
\\ \textbf{Результат} 1234567890864197532
\\
\\ \textbf{Тест 3}
\\ \textbf{Первое число} 1234567890987654321
\\ \textbf{Знак операции} *
\\ \textbf{Второе число} 123456789
\\ \textbf{Результат} 152415787623837841112635269
\\
\\ \textbf{Тест 4}
\\ \textbf{Первое число} 1234567890987654321
\\ \textbf{Знак операции} /
\\ \textbf{Второе число} 123456789
\\ \textbf{Результат} 10000000008

\subsection{Производительность}
\newpage

\section{Заключение}
Этот алгоритм был создан для работы с длинными числами, на случай, если нужно будет посчитать что-либо. Также этот алгоритм можно использовать для большой точности, например, как для подсчета вероятности.
\newpage
\bibliographystyle{alpha}
\bibliography{sample}

\begin{flushleft}
\\
1. \url{https://brestprog.by/topics/longarithmetics/}\\
2. \url{http://e-maxx.ru/algo/big_integer}\\
3. \url{http://comp-science.narod.ru/DL-AR/okulov.htm}\\
4. \url{https://habr.com/ru/post/172285/}\\
5. \url{http://cppalgo.blogspot.com/2010/05/blog-post.html}\\
6. \url{http://cppalgo.blogspot.com/2010/08/div-mod_29.html}\\
7. \url{https://infourok.ru/dlinnaya-arifmetika-na-c-opisanie-modeli-realizaciya-zadachi-1959820.html}\\
8. \url{http://cppstudio.com/en/post/5036/}\\
9. \url{https://readera.org/using-long-arithmetic-in-c-programming-language}\\
10. \url{https://en.wikipedia.org/wiki/Arbitrary-precision_arithmetic}\\
11. \url{http://zonakoda.ru/dlinnaya-arifmetika-vsyo-o-biblioteke-big_int.html}\\
12. \url{https://inf.1sept.ru/2000/1/art/okul1.htm}\\
13. \url{https://yougame.biz/threads/199672/}\\
14. \url{https://forum.antichat.com/threads/50392/}\\
15. \url{https://present5.com/arifmetika-mnogokratnoj-tochnosti-dlinnaya-arifmetika-biryukov-s-v/}\\
16. \url{https://itnan.ru/post.php?c=1&p=578718}\\
17. \url{https://revolution.allbest.ru/programming/00680612_0.html}\\
18. \url{https://www.sites.google.com/site/algoritmyprogramm/c/dlinnye-cisla-v-razrabotke}\\
19. \url{https://www.geeksforgeeks.org/longest-arithmetic-progression-dp-35/}\\
20. \url{https://www.sanfoundry.com/dynamic-programming-solutions-longest-arithmetic-progression-problem/}\\
21. \url{https://www.codingninjas.com/codestudio/library/longest-arithmetic}\\
22. \url{https://github.774.gs/AngelicosPhosphoros/LongArithmeticsCPP}\\
23. \url{https://software-testing.com/topic/597880/long-arithmetic-c/2}\\
24. \url{http://bb3x.ru/blog/dlinnaya-arifmetika-ot-microsoft/}\\
25. \url{https://progaem.forum2x2.ru/t93-topic}\\
26. \url{https://fenlin.ru/video/i1uHzncfVr0}\\
27. \url{https://pro-prof.com/forums/topic/long-integer-arithmetic-cplusplus}\\
\end{flushleft}

\end{document}