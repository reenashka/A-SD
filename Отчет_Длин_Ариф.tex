\documentclass[a4paper]{article}
\usepackage[14pt]{extsizes} % для того чтобы задать нестандартный 14-ый размер шрифта
\usepackage[utf8]{inputenc}
\usepackage[russian]{babel}
\usepackage{setspace,amsmath}
\usepackage[left=20mm, top=15mm, right=15mm, bottom=15mm, nohead, footskip=10mm]{geometry} 
 
\begin{document} 
 
\begin{center}
\hfill \break
\large{МИНИСТЕРСТВО НАУКИ И ВЫСШЕГО ОБРАЗОВАНИЯ РОССИЙСКОЙ ФЕДЕРАЦИИ}\\
\footnotesize{Федеральное государственное автономное образовательное учреждение высшего образования}\\ 
\footnotesize{«Дальневосточный федеральный университет»}\\
\small{\textbf{(ДВФУ)}}\\
\hfill \break
\normalsize{ИНСТИТУТ МАТЕМАТИКИ И КОМПЬЮТЕРНЫХ ТЕХНОЛОГИЙ}\\
 \hfill \break
\normalsize{Департамент математического и компьютерного моделирования}\\
\hfill\break
\hfill \break
\hfill \break
\hfill \break
\large{Эффективная длинная арифметика}\\
\hfill \break
\hfill \break
\hfill \break
\normalsize{Доклад\\
\hfill \break
Направление подготовки 09.03.03 Прикладная информатика\\
\hfill \break
Профиль «Приладная информатика в компьютерном дизайне»}\\
\hfill \break
\hfill \break
\end{center}
 
\normalsize{} \hfill \break
\hfill \break
 
\normalsize{ 
\begin{tabular}{cccc}
Обучающийся & \underline{\hspace{3cm}} & &И.С. Щербак \\\\
Руководитель & \underline{\hspace{3cm}}& доцент ИМКТ &А.С. Кленин \\\\
\end{tabular}
}\\
\hfill \break
\hfill \break
\begin{center} Владивосток 2022 \end{center}
\thispagestyle{empty} 
 
\newpage
     
    \tableofcontents 
\newpage

\section{Введение}
\subsection{О длинной информации}
Длинная арифметика - набор алгоритмов для поразрядной работы с целыми числами произвольной длины. Она применяется как с относительно небольшими целыми числами, превышающими ограничения типа long long в несколько раз, так и с по-настоящему большими целыми числами (чаще всего до 10100000.). Для работы с “длинными” целыми числами их разбивают на разряды.
\\Многие языки (Java, Ruby, Python) имеют встроенную поддержку длинной арифметики, что в разы может сократить время написания программы.
\subsection{История}
Первый бизнес-компьютер IBM 702 (цифровой компьютер на базе лапвовых компьютеров первого поколения) в середине 1950-х годов полностью реализовывал целочисленную арифметику в аппаратных срествах для строк цифр любой длинны от 1 до 511. Самая раняя широко распрастранненая программная реализация арифметики произвольной точности была реализована в Maclisp. 
\\Рання широко распространненная реализация была доступна через IMB 1620 (1959-1970 годов). Эта машина была десятичной, которая использовала дискретные транзисторы, но имела аппаратное обеспечение (которое использовало таблицы поиска) для выполнения целочисленной арифметики над строками цифр длиной от двух до любой доступной памяти. Для арифметики с плавающей точкой мантиса была ограничена сотней цифр или меньше, а показатель степени был ограничен только двумя цифрами. 
\newpage

\section{Описание}
\subsection{Начальные данные}
На вход подеается строка с целыми числами и оператором. Строка может включать в себя 4294967295 символов. 
\\В файле находится строка с данными: <операнд1>, <операция>, <операнд2>.
\newpage

\section{Реализация}
\subsection{Сложение}
\\Рассмотрим арифметическую операцию сложения, применяемую в длинной арифметике.
\\Для начала смотрим на знак чисел. Если оба операнда положительные, то складываются последние разряды двух чисел. И если их сумма больше десяти, то эту сумму делим на десять и переносим это значение к следующему разряду.
\\Если же одно из этих операндов отрицательное, то сначала отбрасываются знаки, из большего вычитается меньшее, и в зависимости от того, какое значение имело большее число, ставится конечный знак.
\\Программа работает следующим образом: происходит сложение от младших разрядов к старшим.
\\
\\ \textbf{Пример}
\\ \textbf{Первое число} 1234567890987654321
\\ \textbf{Второе число} 123456789
\\ \textbf{Результат} 1234567891111111110
\\
\\ \textbf{Пример}
\\ \textbf{Первое число} 1234567890987654321
\\ \textbf{Второе число} -123456789
\\ \textbf{Результат} 1234567890864197532
\\
\\ \textbf{Пример}
\\ \textbf{Первое число} -1234567890987654321
\\ \textbf{Второе число} 123456789
\\ \textbf{Результат} -1234567890864197532

\subsection{Вычитание}.
\\Рассмотрим арифметическую операцию вычитания, применяемую в длинной арифметике.
\\Для начала смотрим на знак чисел. Если оба операнда положительные, то вычитаем последние разряды двух чисел. И если цифра большего числа меньше цифры меньшего числа, то нужно занять единицу у предыдущего разряда большего числа.
\\Если же одно из этих операндов отрицательное, то сначала отбрасываются знаки, большее складывается с меньшим, и в зависимости от того, какое значение имело большее число, ставится конечный знак.
\\Программа работает следующим образом: происходит вычитание от младших разрядов к старшим.
\\
\\ \textbf{Пример}
\\ \textbf{Первое число} 1234567890987654321
\\ \textbf{Второе число} 123456789
\\ \textbf{Результат} 1234567890864197532
\\
\\ \textbf{Пример}
\\ \textbf{Первое число} 1234567890987654321
\\ \textbf{Второе число} -123456789
\\ \textbf{Результат} 1234567891111111110
\\
\\ \textbf{Пример}
\\ \textbf{Первое число} -1234567890987654321
\\ \textbf{Второе число} 123456789
\\ \textbf{Результат} -1234567891111111110

\subsection{Умножение}
\\Рассмотрим арифметическую операцию умножение, применяемую в длинной арифметике.
\\Умножение работает следующим образом: каждый разряд первого числа умножается на каждый разряд второго числа. При умножении разряда i на разряд j добавим результат к разряду i+j произведения и перенос.
\\Далее смотрим на знаки чисел. Если оба операнда положительные или отрицательные, то результат умножения будет положительным. Если первое или второе число отрицательное, то результат умножения будет отрицательным.
\\
\\ \textbf{Пример}
\\ \textbf{Первое число} 1234567890987654321
\\ \textbf{Второе число} 123456789
\\ \textbf{Результат} 152415787623837841112635269
\\
\\ \textbf{Пример}
\\ \textbf{Первое число} 1234567890987654321
\\ \textbf{Второе число} -123456789
\\ \textbf{Результат} -152415787623837841112635269
\\
\\ \textbf{Пример}
\\ \textbf{Первое число} -1234567890987654321
\\ \textbf{Второе число} 123456789
\\ \textbf{Результат} -152415787623837841112635269

\subsection{Деление}
\\Рассмотрим арифметическую операцию деление, применяемую в длинной арифметике.
\\Для начала сравниваем два числа. Если первый операнд больше второго, то из первого операнда вычитается столько раз, сколько в него помещается второе.
\\Далее смотрим на знаки чисел. Если оба операнда положительные или отрицательные, то результат деления будет положительным. Если первое или второе число отрицательное, то результат деления будет отрицательным.
\\ \textbf{Пример}
\\ \textbf{Первое число} 1234567890987654321
\\ \textbf{Второе число} 123456789
\\ \textbf{Результат}10000000008
\\
\\ \textbf{Пример}
\\ \textbf{Первое число} 1234567890987654321
\\ \textbf{Второе число} -123456789
\\ \textbf{Результат} -10000000009
\\
\\ \textbf{Пример}
\\ \textbf{Первое число} -1234567890987654321
\\ \textbf{Второе число} 123456789
\\ \textbf{Результат} -10000000009

\newpage

\section{Формальная постановка задачи}
\\В данной работе требуется:
\\1. Изучить алгоритм по литературным исочникам и описать его в форме научного доклада
\\2. Реализовать алгоритм "Длинная арифметика" с использованием языка программирования С++.
\\3. Длинна строки не должна превышать 4294967295 символов.
\\4. Выполнить исследование данного алгоритма на время
\newpage

\section{Исследование}
\\Данный алгоритм будет исследован на время операций.
\\Время считается в микросекундах.
\subsection{Сложение}
\begin{center}
\begin{tabular}{ |c|c|c|c|c|c|}
\hline
Разрядность & 100 & 200 & 300 & 400 & 500 \\ 
\hline
50 & 1 & 1 & 4 & 6 & 2\\  
100 & 4 & 1 & 2 & 2 & 2\\  
150 & 4 & 1 & 2 & 2 & 2\\
200 & 1 & 1 & 1 & 2 & 2\\  
250 & 4 & 1 & 1 & 2 & 2\\
\hline
\end{tabular}
\end{center}

\subsection{Вычитание}
\begin{center}
\begin{tabular}{ |c|c|c|c|c|c|}
\hline
Разрядность & 100 & 200 & 300 & 400 & 500 \\ 
\hline
50 & 1 & 0 & 1 & 5 & 2\\  
100 & 1 & 0 & 1 & 1 & 5\\  
150 & 1 & 1 & 1 & 1 & 1\\
200 & 3 & 1 & 1 & 1 & 2\\  
250 & 0 & 1 & 1 & 1 & 1\\
\hline
\end{tabular}
\end{center}

\subsection{Произведение}
\begin{center}
\begin{tabular}{ |c|c|c|c|c|c|}
\hline
Разрядность & 100 & 200 & 300 & 400 & 500 \\ 
\hline
50 & 35 & 46 & 66 & 89 & 177\\  
100 & 115 & 114 & 133 & 234 & 316\\  
150 & 114 & 149 & 331 & 466 & 482\\
200 & 126 & 249 & 298 & 584 & 748\\  
250 & 273 & 316 & 458 & 676 & 940\\
\hline
\end{tabular}
\end{center}

\subsection{Деление}
\begin{center}
\begin{tabular}{ |c|c|c|c|c|c|}
\hline
Разрядность & 100 & 200 & 300 & 400 & 500 \\ 
\hline
50 & cell5 & cell6 & cell3 & cell3 & cell3\\  
100 & cell5 & cell6 & cell3 & cell3 & cell3\\  
150 & cell8 & cell9 & cell3 & cell3 & cell3\\
200 & cell5 & cell6 & cell3 & cell3 & cell3\\  
250 & cell8 & cell9 & cell3 & cell3 & cell3\\
\hline
\end{tabular}
\end{center}
\newpage

\section{Тестирование}
\\Данные берутся с файлов.
\\
\\На один тест приходится три файла. Первый файл - \textbf{номер-теста.in}, второй - \textbf{номер-теста.out}, третий - \textbf{номер-теста.ans}.
\\
\\В файле - \textbf{номер-теста.in} лежат Число1, Операнд, Число2.
\\В файле - \textbf{номер-теста.out} лежит результат примера с файла \textbf{номер-теста.in} после работы программы.
\\В файле - \textbf{номер-теста.ans} лежит заранее посчитанный результат.
\\
\\Всего тестов: 91.
\newpage

\section{Заключение}
Этот алгоритм был создан для работы с длинными числами, на случай, если нужно будет посчитать что-либо. Также этот алгоритм можно использовать для большой точности, например, как для подсчета вероятности.
\newpage
\bibliographystyle{alpha}
\bibliography{sample}

\begin{flushleft}
\\
1. \url{https://brestprog.by/topics/longarithmetics/}\\
2. \url{http://e-maxx.ru/algo/big_integer}\\
3. \url{http://comp-science.narod.ru/DL-AR/okulov.htm}\\
4. \url{https://habr.com/ru/post/172285/}\\
5. \url{http://cppalgo.blogspot.com/2010/05/blog-post.html}\\
6. \url{http://cppalgo.blogspot.com/2010/08/div-mod_29.html}\\
7. \url{https://infourok.ru/dlinnaya-arifmetika-na-c-opisanie-modeli-realizaciya-zadachi-1959820.html}\\
8. \url{http://cppstudio.com/en/post/5036/}\\
9. \url{https://readera.org/using-long-arithmetic-in-c-programming-language}\\
10. \url{https://en.wikipedia.org/wiki/Arbitrary-precision_arithmetic}\\
11. \url{http://zonakoda.ru/dlinnaya-arifmetika-vsyo-o-biblioteke-big_int.html}\\
12. \url{https://inf.1sept.ru/2000/1/art/okul1.htm}\\
13. \url{https://yougame.biz/threads/199672/}\\
14. \url{https://forum.antichat.com/threads/50392/}\\
15. \url{https://present5.com/arifmetika-mnogokratnoj-tochnosti-dlinnaya-arifmetika-biryukov-s-v/}\\
16. \url{https://itnan.ru/post.php?c=1&p=578718}\\
17. \url{https://revolution.allbest.ru/programming/00680612_0.html}\\
18. \url{https://www.sites.google.com/site/algoritmyprogramm/c/dlinnye-cisla-v-razrabotke}\\
19. \url{https://www.geeksforgeeks.org/longest-arithmetic-progression-dp-35/}\\
20. \url{https://www.sanfoundry.com/dynamic-programming-solutions-longest-arithmetic-progression-problem/}\\
21. \url{https://www.codingninjas.com/codestudio/library/longest-arithmetic}\\
22. \url{https://github.774.gs/AngelicosPhosphoros/LongArithmeticsCPP}\\
23. \url{https://software-testing.com/topic/597880/long-arithmetic-c/2}\\
24. \url{http://bb3x.ru/blog/dlinnaya-arifmetika-ot-microsoft/}\\
25. \url{https://progaem.forum2x2.ru/t93-topic}\\
26. \url{https://fenlin.ru/video/i1uHzncfVr0}\\
27. \url{https://pro-prof.com/forums/topic/long-integer-arithmetic-cplusplus}\\
\end{flushleft}

\end{document}